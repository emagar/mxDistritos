\documentclass[letter,12pt]{article}
\usepackage[letterpaper,right=1in,left=1in,top=1in,bottom=1in]{geometry}
\usepackage{setspace}

\usepackage[utf8]{inputenc}   % allows input of special characters from keyboard (input encoding)
\usepackage[T1]{fontenc}      % what fonts to use when printing characters       (output encoding)
\usepackage{amsmath}          % facilitates writing math formulas and improves the typographical quality of their output
\usepackage[hyphens]{url}     % adds line breaks to long urls
\usepackage[pdftex]{graphicx} % enhanced support for graphics
\usepackage{tikz}             % Easier syntax to draw pgf files (invokes pgf automatically)
\usetikzlibrary{arrows}

\usepackage{mathptmx}           % set font type to Times
\usepackage[scaled=.90]{helvet} % set font type to Times (Helvetica for some special characters)
\usepackage{courier}            % set font type to Times (Courier for other special characters)

\usepackage[longnamesfirst, sort]{natbib}\bibpunct[]{(}{)}{,}{a}{}{;} % handles biblio and references 

\usepackage{rotating}         % sideway tables and figures that take a full page
\usepackage{caption}          % allows multipage figures and tables with same caption (\ContinuedFloat)

\usepackage{dcolumn}          % needed for apsrtable and stargazer tables from R to compile
\usepackage{arydshln}         % dashed lines in tables (hdashline, cdashline{3-4}, 
                              %see http://tex.stackexchange.com/questions/20140/can-a-table-include-a-horizontal-dashed-line)
                              % must be loaded AFTER dcolumn, 
                              %see http://tex.stackexchange.com/questions/12672/which-tabular-packages-do-which-tasks-and-which-packages-conflict


\newcommand{\mc}{\multicolumn}

%% TO ADD NOTES IN TEXT, PUT % BEFORE THE ONE YOU WANT DISABLED
%\usepackage[disable]{todonotes}                            % no show
\usepackage[colorinlistoftodos, textsize=small]{todonotes} % show notes
\newcommand{\emm}[1]{\todo[color=red!15, inline]{\textbf{Eric:} #1}}

%% \usepackage{xr} % allows cross-ref to other file
%% \externaldocument{urge15appendix}

%% %for submission: sends figs, tables, and footnotes to last pages
%% \RequirePackage[nomarkers,nolists]{endfloat}     % sends tables and figures to the end
%% \RequirePackage{endnotes}                        % turns fn into endnotes; place \listofendnotes where you want 
%%                                                  %the endnotes to appear (it must be after the last endnote).
%% \let\footnote=\endnote
%% \newcommand{\listofendnotes}{
%%    \begingroup
%%    \parindent 0pt
%%    \parskip 2ex
%%    \def\enotesize{\normalsize}
%%    \theendnotes
%%    \endgroup
%% }

%% % for submission: drop page numbers when producing title page
%% \pagenumbering{gobble} % Remove page numbers (and reset to 1)
%% \pagenumbering{arabic}% Arabic page numbers (and reset to 1)

\setcitestyle{citesep={;}}

\usepackage{hyperref}

\begin{document}

\title{The geography of electoral history}
\author{Eric Magar \\ ITAM}
\date{\today}
\maketitle

\noindent The advent of competition in Mexican politics produced a wealth of government data for the analysis of public policy and politics. Data in distributed at the municipal level and smaller geographic units of aggregation (such as census tracts or similar levels), and in some cases at the individual level. It has sprawned fertile areas of new research in education \citep{hoyos.etal.Enlace.2012}, public health \citep{king.etal.segPop.2007,imai.king.nall.segPop.2009}, poverty relief \citep{diaz-estevez-magaloni-Poverty-book.2016,molinar.weldon.1994}, legislative politics \citep{rosas.langston.2011,cantuDesposatoMagar.MxRcv.2014}, and electoral regulation \citep{estevez.magar.rosas.2008}, to name a few.

This paper's focus are vote returns. Electoral data has been distributed for much longer than the information discussed above. It is also better-known and havs received a good deal of attention since Ames' seminal study of the PRI's support bases in municipalities \citep{ames.1970}. This paper describes a repository...

Some of the distributed data is elementary and available elsewhere, such as the number of valid votes cast for parties in congressional races since 1991 in single-member districts, in municipalities, and in sub-municipal units of aggregation.   shares by party and their change since last election) at the municipal and sub-municipal levels. Offers a cross-section time-series of dip fed returns at two levels of aggregation: municipalities and secciones electorales. 

More abstract

from blog

This note presents, discusses, and distributes statistics (available \href{https://github.com/emagar/mxDistritos}{here}) of party performance in Mexico's competitive era. I elaborate two quantities of interest: *voting forecasts* based on recent electoral history and measures of parties' *core support*. The procedure produces summary measures of recent electoral history in relatively small geographic units, municipalities ($N \approx 2500$) and /secciones electorales/ ($N \approx 66000$) throughout Mexico. I apply the methodology to four federal congressional elections between 2009 and 2018 (I will soon apply it to municipal races too), using results since 1994 as historical input.

The note starts by showing the statistics in action to get a glimpse of their descriptive and analytic potential. By summarizing recent electoral history and its geography, the quantities offer a scenic view of a critical aspect of contemporary Mexican politics. 

Later sections offer methodological detail on the estimation of these quantities of interest and are increasingly technical. 

\section{Statistics in action}
Table [[fig:1]] presents two pairs of diagrams for the 2015 (below) and 2018 (above) congressional elections. Each dot represents one municipality, colored according to the winning party, with coordinates in the ternary plot according to the relative votes of the PAN, the PRI, and the left in the federal deputy race (other smaller parties are excluded).\footnote{I must note that, for the left's electoral history (which I arbitrarily call ``Morena'' in the plots and distributed data), I systematically added up the votes of the PRD, PT, and MC up to 2015. I also added Morena's and PES's votes that year. In 2018 the left consisted of Morena, PT, and PES jointly.} 

\begin{figure}
\centering
#+CAPTION: Historic expectations and municipal-level vote in two congressional elections
#+NAME:   fig:1
| file:../assets/img/triplot2018-vhat-mu.png | [[file:../assets/img/triplot2018-v-mu.png]] |
| file:../assets/img/triplot2015-vhat-mu.png | [[file:../assets/img/triplot2015-v-mu.png]] |

\end{figure}


The left side shows the /vote forecasts/. The idea behind this statistic is summarizing the evolution of relative votes in the municipality in five previous elections (2003--2015 in the case of 2018) and using the tendency to project a vote forecast for the current year. Plots in the right side show the actual results observed in both elections.

Three features are noteworthy in 2018. The first is the discrepancy between the left and right plots. Either the model does a poor job forecasting, or 2018 was an extraordinary election. History gave license to expect a comfortable PRI victory, both in the number of municipalities won and in margins of victory. Municipalities outside the dotted bands are won by margins of 15 points or more, and the bulk of secure municipalities are red in the forecast, with Morena in a distant second place. In fact, although a significant number of municipalities migrated towards the PAN, it was Morena who showed a clear advantage. PRI was the underachiever. In contrast, the lower left and right plots reveal fewer differences between them---2015 was a more normal election, the past offering much better grounds to forecast. 

Second, observed municipalities fled the edges and triangle vertices in 2018. Observations in vertices show a party that has no significant challenger. While those on the edges were bipartisan, whether more (inside dotted bands) or less (outside the bands) symmetric. It is also plain in forecasts that only the PAN--Morena edge was expected to be unpopulated. In practice, however, third party vote rarely collapsed to zero, there was much more dis-coordination than in the past. In fact, the intersection of dotted bands appeared denser and more homogeneous in the right than the left diagram. 

Third, the PAN /vs/ left competition was legal tender in 2018. The pattern in competitive municipalities in the last two decades, still visible in the 2015 plots, involved either PRI--PAN or PRI--left rivalries, and rarely PAN--izquierda. It was this pattern that eased electoral alliances between PAN and PRD in sub national races since 2010 that culminated in the Frente they formed in 2018 to nominate a joint presidential candidate. 

# #+CAPTION: Una elección más característica de la partidocracia
# #+NAME:   fig:2
# | file:../assets/img/triplot2015-vhat-mu.png | [[file:../assets/img/triplot2015-v-mu.png]] |


# #+CAPTION: Grano más fino: las secciones
# #+NAME:   fig:3
# | file:../assets/img/triplot2015-v-se.png | [[file:../assets/img/triplot2018-v-se.png]] |

Plots in Table [[fig:4]] report /secciones electorales/ and therefore offer much finer-grained than the previous portraits. They introduce the other quantity of interest in this note: parties' /core support/. The idea behind this statistic is measuring the size of the group that has historically supported the party consistently, in good but also in bad years. 

#+CAPTION: Support core and party performance in 2018
#+NAME:   fig:4
| file:../assets/img/resid-pan-2018-vs-pan-core-se.png | [[file:../assets/img/resid-morena-2018-vs-morena-core-se.png]] | file:../assets/img/resid-pri-2018-vs-pri-core-se.png |

The horizontal axis in each plot measures the size of party core support groups as a proportion of the /sección/'s electorate. The PRI enjoyed a clear edge over the rest of the parties in the period, with sizable cores in /all/ secciones nationwide. The distributions of the PAN and the left, in contrast, appear concentrated towards the zero---they have relatively few secciones with some unconditional support. 

The vertical axis reports the three parties' performance in 2018 (the difference between the observed vote and the forecast). Positive values indicate that the party excelled expectations in the /sección/, negative ones that it under-performed it expectation. The PRI's electoral disaster appears all too clearly in the red plot. There were a few secciones with positive performance, but the density concentrates massively below the horizontal zero line, something that Table [[fig:1]] had hinted. What is truly remarkable is that the dismal performance is directly proportional to the size of the PRI core. President Peña and candidate Meade achieved what seemed, if not impossible, extremely improbable: they alienated the PRI's unconditional voters in 2018. The PAN and the left met expectations in secciones where they have enjoyed with groups of support. Both (especially Morena) over-achieved where they lack important cores, taking away PRI voters. 

The note elaborates how statistics were prepared (replication code can be found [[https://github.com/emagar/mxDistritos/code/elec-data-for-maps.r][here]]).

# [[file:https://github.com/emagar/elecRetrns/raw/master/graph/nytAmloPlusAnayaPlusMeadeNegPenaWon.svg]]

# #+CAPTION: PAN
# #+NAME:   fig:6
# #+ATTR_HTML: style="float:right;"
# #+ATTR_HTML: :width 50%
# [[file:../assets/img/resid-pan-2018-vs-pan-core-se.png]]

* First differences
One common approach to study electoral change is through first differences. Denoting $v_t$ the party's vote share in the municipality or the /sección/, in period $t$, the first difference is simply $d_t = v_t - v_{t-1}$. 

$d_t$ is an intuitive quantity, showing the sign and magnitude of change from one election to the next. But, precisely because it compares pairs of consecutive elections only, it misses more dynamic processes in the units. One example, well documented by electoral sociology, is the regression to the mean \citep{campbell.1991,segovia.1979}. Its detection requires observing the unit through at least three consecutive periods to verify contrary signs in $d_{t+1}$ and $d_t$. The study of secular change in the Mexican party system in the last quarter of century calls for deeper historical perspective. 

(First differences appear in the fields ~d.pan~, ~d.pri~, and ~d.morena~ in the distributed data.)

* The recent linear tendency
One way of adopting it is with /vote forecasting/ from tendencies discernible in the previous five congressional races \citep{magar.gubCoatMx.2012}. I summarize the central tendency of the recent historical vote by means of linear estimation in time, fitting a straight line for each year analyzed and each party in the municipality or /sección electoral/. 

The slope of the fitted line (the tendency) serves to extrapolate the party's electoral support to the future. For instance, to get the vote share that the recent past predicts for a party in unit $u$ for 2018, I estimate the following equation 

\begin{equation}
v_{ut} = a_u + b_u \times t + \text{error}_{ut}, \; t = 2003, \ldots, 2015
\end{equation}\label{ts-eq}

that I then use to predict $\hat{v}_{u2018} = \hat{a}_u + \hat{b}_u \times 2018$. This is an out-of-sample prediction of the party's vote share, it can be compared to the party's actual vote share in 2018 to gauge whether or not the unit approximates the historical record. For the 2015 forecast the sample shifts one period to become $t = 2000, \ldots, 2012$, and so on an so forth for previous years. I distribute forecasts for 2009, 2012, 2015, and 2018, which involved fitting about 10 thousand municipal regressions and more than 250 thousand sección-level.

(Vote forecasts appear in the fields ~vhat.pan~, ~vhat.pri~, and ~vhat.morena~ of the distributed data.)

* The party's support core
The other historical statistic is the parties' core support in the unit. Its definition stems from classifying voters in three categories: (1) support groups, that in the past have consistently supported the party; (2) opposition groups, that have consistently supported another party; and (3) swing groups, that have neither consistently supported nor consistently opposed the party \citep{cox.mccubbins.1986}. The party core consists of the support groups. 

I use the procedure by Díaz Cayeros /et al/. (2016) to estimate this core. If $\bar{v}_t$ denotes the party's mean support across all units in period $t$,\footnote{The analyzed unit $u$ should be dropped from period $t$'s mean in order to not include the dependent variable in the right side of equation \ref{ts-eq}. I do not drop it due to the large number of municipal or seccional units (each contributing a fraction to the mean) and the use of vote shares (so large units are watered down): this refinement's impact should be negligible in each mean value.} for each party in each unit I fit

\begin{equation}
v_{ut} = \alpha_u + \beta_u \times \bar{v}_t + \text{error}_{ut}, \; t = 1994, \ldots, 2018.
\end{equation}\label{alpha-eq}

$\beta_u$ measures the effect that national party tides have on the party's vote in unit $u$. For instance, $\hat{\beta}_u=1$ estimates that for every percentage point that the party wins or loses nationally in year $t$, it also wins or loses one percentage point in the unit; $\hat{\beta}_u=0$, on the other hand, would indicate the unit's full isolation form national swings. It is therefore a measure of party volatility in the municipality or /sección/ (analogous to \href{https://www.investopedia.com/terms/v/volatility.asp}{"beta volatility"} in the financial literature). 

The $\alpha$ coefficient estimates the core size: expected support in unit $u$ in the hypothetical case that the party receives no vote at the national level. For instance, $\hat{\alpha}=.4$ would indicate that, in the starkest of scenarios, 40% of the electorate in the municipality is unconditional to the party---which would indeed be a core of substantial size.

A critique that can be anticipated towards this measure of the party's support core is its extreme counter-factual nature \citep{king.zeng.counterfactuals2006pa}. It deserves rigorous scrutiny, something I plan doing in the near future. 

(The parties' core support appears in the fields ~alphahat.pan~, ~alphahat.pri~, and ~alphahat.morena~ of the distributed data. Party volatility in ~betahat.pan~, ~betahat.pri~, and ~betahat.morena~.)

* Compositional variables
I close with an important feature of the model specifications, associated with the \emph{compositional} nature of electoral returns. Compositional variable are quantitative descriptions of the parts of a whole. They therefore have two characteristics: (1) they are proportions that (2) add up to unity.\footnote{Formally, the compositional are random variables subject to two constraints: 
$0 \leq v_p \leq 1 \; \forall \; p \in P \; \; and \; \; \sum_P v_p = 1.$.} 

When estimating parties separately, the challenge of equations 1 and 2 is to avoid forecasting vote shares less than zero or greater than one; and that the sum of party forecasts equals 1. To achieve this, \citet{aitchison.1986} proposes substituting vote shares by log-ratios in the analysis. Arbitrarily setting the PRI as the reference party, define party $p$'s vote relative to the PRI as 

$r_p = \frac{v_p}{v_{\text{pri}}}.$

A value $r_p=1$ would indicate a tie between the party and the PRI, while $r_p>1$ that it finished ahead of the PRI in the proportion that the value reveals. 

Thus, equation 1 is re-specified as

$\ln r_{put} = a + b \times t + \text{error}$

y equation 2 as

$\ln r_{put} = \alpha + \beta \times \bar{r}_{pt} + \text{error}.$

Applying the natural logarithm attenuates the effect of extreme values of the regressor on the dependent variable, similar as a logit regression would. Models fitting was done with ordinary least squares. 

Coefficient estimates requires transformation to collect party vote shares. Illustrating with a three-party caste, it is trivial that 

\begin{equation}
\hat{v}_p = \frac{\hat{r}_p}{1 + \hat{r}_{\text{pan}} + \hat{r}_{\text{morena}}} \; \text{and} \;
\hat{v}_{\text{pri}} = \frac{1}{1 + \hat{r}_{\text{pan}} + \hat{r}_{\text{morena}}.}
\end{equation}

%% \begin{equation}
%% \begin{split}
%% v_{\text{pri}} + v_{\text{pan}} + v_{\text{morena}} & = 1 \\
%% v_{\text{pri}} & = 1 - v_{\text{pan}} - v_{\text{morena}} \\
%% 1 & = \frac{1}{v_{\text{pri}}} - \frac{v_{\text{pan}}}{v_{\text{pri}}} - \frac{v_{\text{morena}}}{v_{\text{pri}}} \\
%% \end{split}
%% \end{equation}

These are the quantities that the distributed data report. 


# Fácil de implementar en R:
# a = voto partido unidad 1
# A = voto efec unidad 1
# N = 3 unidades
#
# tengo
# v.bar = 1/3 * (a/A + b/B + c/C)
#
# quiero
# v.bar.sin.aA = 1/2 * (b/B + c/C)
#
# hago
# 1/3 * (a/A + b/B + c/C) =      v.bar
#        a/A + b/B + c/C  =  3 * v.bar
#              b/B + c/C  =  3 * v.bar - a/A
#       1/2 * (b/B + c/C) = (3 * v.bar - a/A) * 1/2 = v.bar.sin.aA
#
# v.bar.sin.aA = (N * v.bar - a/A) * 1/(N-1)



\bibliographystyle{apsr}
\bibliography{/home/eric/Dropbox/mydocs/magar}


\end{document}
